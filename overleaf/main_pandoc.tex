% !TEX TS-program = xelatex
% !BIB program = bibtex
% !TEX encoding = UTF-8 Unicode
% Options for packages loaded elsewhere
\PassOptionsToPackage{unicode}{hyperref}
\PassOptionsToPackage{hyphens}{url}

\documentclass[
  twoside,
  openright,
  degree    = master,               % degree = master | doctor
  language  = english,              % language = chinese | english
  fontset   = overleaf,             % fontset = default | template | system | overleaf
  watermark = true,                 % watermark = true | false
  doi       = true,                 % doi = true | false
]{ntuthesis}

% !TeX root = ./main.tex

% --------------------------------------------------
% 資訊設定(Information Configs)
% --------------------------------------------------

\ntusetup{
  university*   = {National Taiwan University},
  university    = {國立臺灣大學},
  college       = {社會科學院},
  college*      = {College of Social Science},
  institute     = {國家發展研究所},
  institute*    = {Graduate Institute of National Development},
  title         = {淨零轉型下臺灣能源貧窮家戶特徵分析},
  title*        = {Toward Net-Zero: Examining Household Characteristics
and the Risk of Energy Poverty under Taiwan's Energy Transition},
  author        = {吳齊穎},
  author*       = {Chiying Wu},
  ID            = {R10341027},
  advisor       = {林竣達},
  advisor*      = {Jiunda Lin},
  date          = {2024-12-09},
  oral-date     = {2025-01-07},
  DOI           = {10.5566/NTU2024XXXXX},
  keywords      = {淨零轉型, 能源貧窮, 低收入高支出, 家庭收支調查,
多層次分析},
  keywords*     = {net-zero transition, energy poverty, LIHC, SFIE,
GLMM}
}

% \usepackage[sort&compress]{natbib}      % 參考文獻
\usepackage{amsmath, amsthm, amssymb}   % 數學環境
% \usepackage{ulem, CJKulem}              % 下劃線、雙下劃線與波浪紋效果
\usepackage{booktabs}                   % 改善表格設置
\usepackage{multirow}                   % 合併儲存格
\usepackage{diagbox}                    % 插入表格反斜線
\usepackage{array}                      % 調整表格高度
\usepackage{longtable}                  % 支援跨頁長表格
\usepackage{paralist}                   % 列表環境

%%%%%%%%%%%% Pandoc Preamble %%%%%%%%%%%%%%
\usepackage{lmodern}
\usepackage{iftex}
\ifPDFTeX
  \usepackage[T1]{fontenc}
  \usepackage[utf8]{inputenc}
  \usepackage{textcomp} % provide euro and other symbols
\else % if luatex or xetex
  \usepackage{unicode-math}
  \defaultfontfeatures{Scale=MatchLowercase}
  \defaultfontfeatures[\rmfamily]{Ligatures=TeX,Scale=1}
\fi
% Use upquote if available, for straight quotes in verbatim environments
\IfFileExists{upquote.sty}{\usepackage{upquote}}{}
\IfFileExists{microtype.sty}{% use microtype if available
  \usepackage[]{microtype}
  \UseMicrotypeSet[protrusion]{basicmath} % disable protrusion for tt fonts
}{}
% \makeatletter
% \@ifundefined{KOMAClassName}{% if non-KOMA class
%   \IfFileExists{parskip.sty}{%
%     \usepackage{parskip}
%   }{% else
%     \setlength{\parindent}{0pt}
%     \setlength{\parskip}{6pt plus 2pt minus 1pt}}
% }{% if KOMA class
%   \KOMAoptions{parskip=half}}
% \makeatother
\usepackage{xcolor}
\IfFileExists{xurl.sty}{\usepackage{xurl}}{} % add URL line breaks if available
\IfFileExists{bookmark.sty}{\usepackage{bookmark}}{\usepackage{hyperref}}
\hypersetup{
  hidelinks,
  pdfcreator={LaTeX via pandoc}}
\urlstyle{same} % disable monospaced font for URLs
\usepackage{color}
\usepackage{fancyvrb}
\newcommand{\VerbBar}{|}
\newcommand{\VERB}{\Verb[commandchars=\\\{\}]}
\DefineVerbatimEnvironment{Highlighting}{Verbatim}{commandchars=\\\{\}}
% Add ',fontsize=\small' for more characters per line
% \newenvironment{Shaded}{}{}
\usepackage{framed}
\definecolor{shadecolor}{RGB}{248,248,248}
\newenvironment{Shaded}{\begin{snugshade}}{\end{snugshade}}
\newcommand{\AlertTok}[1]{\textcolor[rgb]{1.00,0.00,0.00}{\textbf{#1}}}
\newcommand{\AnnotationTok}[1]{\textcolor[rgb]{0.38,0.63,0.69}{\textbf{\textit{#1}}}}
\newcommand{\AttributeTok}[1]{\textcolor[rgb]{0.49,0.56,0.16}{#1}}
\newcommand{\BaseNTok}[1]{\textcolor[rgb]{0.25,0.63,0.44}{#1}}
\newcommand{\BuiltInTok}[1]{#1}
\newcommand{\CharTok}[1]{\textcolor[rgb]{0.25,0.44,0.63}{#1}}
\newcommand{\CommentTok}[1]{\textcolor[rgb]{0.38,0.63,0.69}{\textit{#1}}}
\newcommand{\CommentVarTok}[1]{\textcolor[rgb]{0.38,0.63,0.69}{\textbf{\textit{#1}}}}
\newcommand{\ConstantTok}[1]{\textcolor[rgb]{0.53,0.00,0.00}{#1}}
\newcommand{\ControlFlowTok}[1]{\textcolor[rgb]{0.00,0.44,0.13}{\textbf{#1}}}
\newcommand{\DataTypeTok}[1]{\textcolor[rgb]{0.56,0.13,0.00}{#1}}
\newcommand{\DecValTok}[1]{\textcolor[rgb]{0.25,0.63,0.44}{#1}}
\newcommand{\DocumentationTok}[1]{\textcolor[rgb]{0.73,0.13,0.13}{\textit{#1}}}
\newcommand{\ErrorTok}[1]{\textcolor[rgb]{1.00,0.00,0.00}{\textbf{#1}}}
\newcommand{\ExtensionTok}[1]{#1}
\newcommand{\FloatTok}[1]{\textcolor[rgb]{0.25,0.63,0.44}{#1}}
\newcommand{\FunctionTok}[1]{\textcolor[rgb]{0.02,0.16,0.49}{#1}}
\newcommand{\ImportTok}[1]{#1}
\newcommand{\InformationTok}[1]{\textcolor[rgb]{0.38,0.63,0.69}{\textbf{\textit{#1}}}}
\newcommand{\KeywordTok}[1]{\textcolor[rgb]{0.00,0.44,0.13}{\textbf{#1}}}
\newcommand{\NormalTok}[1]{#1}
\newcommand{\OperatorTok}[1]{\textcolor[rgb]{0.40,0.40,0.40}{#1}}
\newcommand{\OtherTok}[1]{\textcolor[rgb]{0.00,0.44,0.13}{#1}}
\newcommand{\PreprocessorTok}[1]{\textcolor[rgb]{0.74,0.48,0.00}{#1}}
\newcommand{\RegionMarkerTok}[1]{#1}
\newcommand{\SpecialCharTok}[1]{\textcolor[rgb]{0.25,0.44,0.63}{#1}}
\newcommand{\SpecialStringTok}[1]{\textcolor[rgb]{0.73,0.40,0.53}{#1}}
\newcommand{\StringTok}[1]{\textcolor[rgb]{0.25,0.44,0.63}{#1}}
\newcommand{\VariableTok}[1]{\textcolor[rgb]{0.10,0.09,0.49}{#1}}
\newcommand{\VerbatimStringTok}[1]{\textcolor[rgb]{0.25,0.44,0.63}{#1}}
\newcommand{\WarningTok}[1]{\textcolor[rgb]{0.38,0.63,0.69}{\textbf{\textit{#1}}}}
\usepackage{multirow}
\usepackage{calc} % for calculating minipage widths
% Correct order of tables after \paragraph or \subparagraph
\usepackage{etoolbox}
\makeatletter
\patchcmd\longtable{\par}{\if@noskipsec\mbox{}\fi\par}{}{}
\makeatother
% Allow footnotes in longtable head/foot
\IfFileExists{footnotehyper.sty}{\usepackage{footnotehyper}}{\usepackage{footnote}}
\makesavenoteenv{longtable}
% Make links footnotes instead of hotlinks:
\DeclareRobustCommand{\href}[2]{#2\footnote{\url{#1}}}
\setlength{\emergencystretch}{3em} % prevent overfull lines
\providecommand{\tightlist}{%
  \setlength{\itemsep}{0pt}\setlength{\parskip}{0pt}}
\setcounter{secnumdepth}{5}
\newlength{\cslhangindent}
\setlength{\cslhangindent}{1.5em}
\newlength{\csllabelwidth}
\setlength{\csllabelwidth}{3em}
\newlength{\cslentryspacingunit} % times entry-spacing
\setlength{\cslentryspacingunit}{\parskip}
\newenvironment{CSLReferences}[2] % #1 hanging-ident, #2 entry spacing
 {% don't indent paragraphs
  \setlength{\parindent}{0pt}
  % turn on hanging indent if param 1 is 1
  \ifodd #1
  \let\oldpar\par
  \def\par{\hangindent=\cslhangindent\oldpar}
  \fi
  % set entry spacing
  \setlength{\parskip}{#2\cslentryspacingunit}
 }%
 {}
\newcommand{\CSLBlock}[1]{#1\hfill\break}
\newcommand{\CSLLeftMargin}[1]{\parbox[t]{\csllabelwidth}{#1}}
\newcommand{\CSLRightInline}[1]{\parbox[t]{\linewidth - \csllabelwidth}{#1}\break}
\newcommand{\CSLIndent}[1]{\hspace{\cslhangindent}#1}
\makeatletter
% \@ifpackageloaded{subfig}{}{\usepackage{subfig}}
% \@ifpackageloaded{caption}{}{\usepackage{caption}}
\captionsetup[subfloat]{margin=0.5em}
\newcounter{pandoccrossref@subfigures@footnote@counter}
\newenvironment{pandoccrossrefsubfigures}{%
\setcounter{pandoccrossref@subfigures@footnote@counter}{0}
\begin{figure}\centering%
\gdef\global@pandoccrossref@subfigures@footnotes{}%
\DeclareRobustCommand{\footnote}[1]{\footnotemark%
\stepcounter{pandoccrossref@subfigures@footnote@counter}%
\ifx\global@pandoccrossref@subfigures@footnotes\empty%
\gdef\global@pandoccrossref@subfigures@footnotes{{##1}}%
\else%
\g@addto@macro\global@pandoccrossref@subfigures@footnotes{, {##1}}%
\fi}}%
{\end{figure}%
\addtocounter{footnote}{-\value{pandoccrossref@subfigures@footnote@counter}}
\@for\f:=\global@pandoccrossref@subfigures@footnotes\do{\stepcounter{footnote}\footnotetext{\f}}%
\gdef\global@pandoccrossref@subfigures@footnotes{}}
\@ifpackageloaded{float}{}{\usepackage{float}}
\floatstyle{ruled}
\@ifundefined{c@chapter}{\newfloat{codelisting}{h}{lop}}{\newfloat{codelisting}{h}{lop}[chapter]}
\floatname{codelisting}{程式碼}
\newcommand*\listoflistings{\listof{codelisting}{程式碼目錄}}
\makeatother

% Linguistic alignment
\usepackage{linguex}
\renewcommand{\theExLBr}{}
\renewcommand{\theExRBr}{}
\newcommand{\jdg}[1]{\makebox[0.4em][r]{\normalfont#1\ignorespaces}}
\usepackage{chngcntr}
\ifLuaTeX
  \usepackage{selnolig}  % disable illegal ligatures
\fi

% Set independent linestretch for code chunks
\let\oldShaded=\Shaded
\let\endoldShaded=\endShaded
\renewenvironment{Shaded}{
    \begin{spacing}{1}\begin{oldShaded}
  }
  {
  \end{oldShaded}
  \end{spacing}
  \vspace{0.1cm}
  }
\let\oldLongTable=\longtable
  \let\endoldLongTable=\endlongtable
  \renewenvironment{longtable}{
      \begin{small}\begin{spacing}{1.1}\begin{oldLongTable}
    }
    {
    \end{oldLongTable}\end{spacing}\end{small}
    }
\let\oldTable=\table
  \let\endoldTable=\endtable
  \renewenvironment{table}{
    \begin{small}\begin{spacing}{1.1}\begin{oldTable}
    }
    {
    \end{oldTable}\end{spacing}\end{small}
    }

%%%%%% Pandoc's graphicx config %%%%%%%
% For strikeout in LaTex
\usepackage[normalem]{ulem}
\pdfstringdefDisableCommands{\renewcommand{\sout}{}}
\usepackage{paralist}  % Compact list
%%%%%%%%%%%%%%%%%%%%%%%%%% END Pandoc Preamble %%%%%%%%%%%%%%%%%%%%%%%%%%%%%%%%%%%%%%%%

%%%% Change verbatim env: linestrech, before/after vspace, text color
\makeatletter
  \preto{\@verbatim}{\topsep=0pt \partopsep=5pt }
  \renewcommand\verbatim@font{\normalfont\ttfamily\color{gray}}
\makeatother
\BeforeBeginEnvironment{verbatim}{
% \noindent
% \texttt{\textbf{Output}}
\def\baselinestretch{1}
}

\begin{document}

% Compact list
\renewenvironment{itemize}[1]{
\vspace{0.2cm} \begin{compactitem}#1}
{\end{compactitem} \vspace{0.5cm}}
\renewenvironment{enumerate}[1]{
\vspace{0.2cm}\begin{compactenum}#1}
{\end{compactenum} \vspace{0.5cm}}




% 封面與口試審定
% Cover and Verification Letter
\makecover                          % 論文封面(Cover)
% 口試委員審定書(Verification Letter)
\IfFileExists{./口試委員審定書.pdf}{
    \includepdf[angle=0]{./口試委員審定書.pdf}
    }{\makeverification}

%%%%%%%%%%%%%%%%% Acknowledgement %%%%%%%%%%%%%%%%% 
\begin{acknowledgement}{\enstretch
謝天,謝地。
\par}\end{acknowledgement}
%%%%%%%%%%%%%%%%%%%%%%%%%%%%%%%%%%%%%%%%%%%%%%%%%%% 

%%%%%%%%%%%%%%%%%%%%% Abstract %%%%%%%%%%%%%%%%%%%%
\begin{abstract}{\zhstretch
能源貧窮指涉無法近用現代能源或是不足以負擔其成本。在台灣,此一議題在積極推動能源轉型的背景下受到了廣泛關注。台灣淨零轉型的路徑高度依賴天然氣的使用,
\par}\end{abstract}

\begin{abstract*}{\enstretch
Energy poverty, characterized by a lack of access to modern energy
services or the inability to afford adequate energy consumption,
presents a pressing challenge that influences personal well-being and
societal development. In Taiwan, this issue has gained significant
attention amidst an ambitious energy transition strategy aimed at
reducing coal consumption, phasing out nuclear power, and increasing
reliance on renewable energy and natural gas. However, Taiwan's heavy
dependence on imported energy resources, particularly natural gas,
coupled with limited stockpiling capacity, exposes the country to supply
disruptions and price fluctuations, exacerbating existing energy
inequalities. Low-income and small-sized households in Taiwan often face
higher per capita energy consumption due to inefficient appliances,
which intensifies their energy burden under the progressive tariff
system. Gender disparities further contribute to energy poverty, with
female-headed households being particularly vulnerable. This research
investigates the non-linear relationship between household size and
energy poverty risk, employing an expenditure-based Low-Income High Cost
(LIHC) approach. Using data from Taiwan's 2021 Survey of Family Income
and Expenditure (SFIE), the study utilizes a Generalized Linear Mixed
Model (GLMM) to explore the impact of household head's gender and family
composition on energy poverty. The findings aim to provide nuanced
insights into the complexities of energy poverty in Taiwan, informing
targeted interventions for vulnerable populations.
\par}\end{abstract*}
%%%%%%%%%%%%%%%%%%%%%%%%%%%%%%%%%%%%%%%%%%%%%%%%%%% 

% 生成目錄與符號列表
% Contents of Tables and Denotation
\maketableofcontents                % 目錄(Table of Contents)
\makelistoffigures                  % 圖目錄(List of Figures)
\makelistoftables                   % 表目錄(List of Tables)

\mainmatter

\hypertarget{sec:introduction}{%
\chapter{Introduction}\label{sec:introduction}}

Energy poverty, described as the lack of access to modern energy
services or the inability to afford adequate levels of energy
consumption, is a critical issue affecting both personal development and
societal well-being (González-Eguino, 2015). Addressing energy poverty
aligns with the United Nations' Sustainable Development Goal 7, which
aims to ensure access to affordable, reliable, sustainable, and modern
energy for all by 2030 (Lin et al., 2020). Affordable and accessible
energy services have significant implications across various sectors,
including health, education, and economic development (Akram, 2022; Baum
et al., 2022).

In Taiwan, energy poverty has garnered increasing attention, calling for
tailored approaches to tackle this multifaceted issue. The Taiwanese
government is implementing an ambitious energy transition plan to
mitigate the effects of climate change. This plan focuses on expanding
renewable energy sources, increasing natural gas as a transitional fuel,
reducing coal consumption, and phasing out nuclear power (P.-H. Chen et
al., 2023; National Development Council, 2024). While natural gas offers
several environmental advantages over coal---such as lower greenhouse
gas emissions and fewer air pollutants (Aramillo et al.,
2007)---Taiwan's increasing reliance on natural gas introduces new
challenges.

Taiwan imports nearly all its energy resources, with 99\% of natural gas
being imported(Bureau of Energy, 2024). This heavy dependence, coupled
with limited stockpiling capacity of only 7 to 11 days, creates
significant vulnerabilities(Chien, 2020). Geopolitical tensions,
particularly the Ukraine-Russia conflict, have contributed to rising
energy prices in the residential sector (Taiwan Research Institute,
2023). As Taipower has been selling electricity at a loss, there is a
growing need for energy price adjustments to ensure the sustainability
of the supply system. This dependence makes Taiwan more susceptible to
supply disruptions and price fluctuations, potentially exacerbating
existing energy inequality issues.

From an energy consumption perspective, low-income and small-sized
households in Taiwan may have higher per capita energy consumption
(Huang, 2015). The progressive electricity tariff system adopted by
Taipower can further amplify this burden, as low-income households often
lack the financial means to replace energy-inefficient appliances,
leading to higher expenditures for the same level of energy services
(Su, 2020). Energy price shocks disproportionately affect households at
different income levels, with low-income households being more
vulnerable (Tuttle \& Beatty, 2017).

Income disparities between male-headed and female-headed households
remain significant in Taiwan. According to the 2023 Report on Family
Income and Expenditure(DGBAS, 2024), male-headed households have a
higher average disposable income (1,235,000 NTD) compared to
female-headed households (937,000 NTD). While per capita disposable
income shows no significant difference when adjusting for household
size, energy poverty is experienced based on a per-household basis. This
suggests that female-headed households may be more vulnerable to energy
poverty.

Previous studies in Taiwan have examined residential electricity
consumption patterns and their determinants (Huang, 2015; Su, 2019,
2020). For instance, Huang (2015) employed quantile regression to
analyze how household characteristics affect energy usage, finding that
household size influences electricity consumption across all quantiles.
Su et al.~(2020) extended this focus by incorporating urbanization and
energy poverty into residential electricity demand estimations,
suggesting that smaller-sized households should take more responsibility
in energy conservation efforts.

However, these studies predominantly concentrated on energy consumption,
they often overlooked the broader social justice dimensions encompassed
by energy poverty. This research seeks to highlight the complexity of
energy poverty issues in Taiwan and their multifaceted implications.
Specifically, we propose that for energy poverty risk, not only is there
disparity between male- and female-headed households, but there exists a
non-linear relationship between household size and the risk of falling
into energy poverty. While smaller households may have higher per capita
energy usage due to limited economies of scale, larger households with
more dependents or fewer income recipients may face greater financial
strain in paying energy bills.

To effectively capture the affordability aspect of energy poverty in
Taiwan, this study adopts an expenditure-based method using the
Low-Income High Cost (LIHC) approach. This method identifies households
that are low-income but face disproportionate energy burdens. Utilizing
data from the 2023 Survey of Family Income and Expenditure (SFIE), which
includes comprehensive information on household composition, attributes,
facilities, housing conditions, income, and expenditure (DGBAS, 2024),
we employ a Generalized Linear Mixed Model (GLMM) to analyze the effects
of household head's gender and family composition on the risk of falling
into energy poverty.

Our research contributes to the existing literature by shifting the
focus from analyzing energy consumption patterns to examining the
complex factors influencing energy poverty in Taiwan. By considering
both household head gender and family structure, we aim to provide a
more nuanced understanding of energy poverty, informing policymakers to
develop targeted interventions that address the specific needs of
vulnerable populations. In Taiwan, although energy prices have been kept
at a low and affordable range, leading many to perceive it as an
insignificant issue. However, the disparity still exist, with certain
households being more vulnerable to Net-Zero transition. Common policy
interventions to address climate change often involve raising energy
prices to incentivize reduced electricity consumption. However, such
measures can disproportionately affect these vulnerable households.
Therefore, tailored approaches are necessary to ensure that climate
policies are both effective and equitable.

The remainder of this article will be divided into five parts.
\protect\hyperlink{sec:literaturereview}{Section 2} provides a general
literature review of energy poverty landscape around the world and
hypotheses of this article. \protect\hyperlink{sec:methodology}{Section
3} illustrates the research method and design.
\protect\hyperlink{sec:results}{Section 4} will be presenting the
empirical findings. The \protect\hyperlink{sec:conclusion}{final
section} discusses the contributions and implications of this study.

\hypertarget{sec:literaturereview}{%
\chapter{Literature Review}\label{sec:literaturereview}}

Energy poverty lacks a universal definition(Kumar et al., 2019), whereas
it may not be entirely undesirable given its deep connection to
historical and geographical contexts(Bouzarovski \& Simcock, 2017). With
respect to different origins, `fuel poverty' and `energy poverty' are
often used interchangeably, where ``fuel poverty'' traditionally
describes households in developed economies unable to afford adequate
heating, and ``energy poverty'' broadly refers to a lack of access to
energy services in less developed countries(Das et al., 2022). The
concept of fuel poverty was formally introduced byBoardman (1991) in her
seminal work on the UK, where the focus was primarily on housing heating
affordability. Over time, however, discussions surrounding fuel poverty
expanded to consider broader forms of energy deprivation beyond just
fuel costs.

As the globalization and marketization of energy markets, along with the
expansion of academic fields, the traditional dichotomy between fuel
poverty and energy poverty has been challenged(Bouzarovski \& Petrova,
2015). Although there are disagreement such asLi et al. (2014) argued
that there are clear geographical and socioeconomic distinctions between
fuel poverty and energy poverty, and some people may suffer from both
simultaneously, but in a scientometric analysis of the evolution of
energy poverty theory,Guevara et al. (2022) found that while the
concepts of fuel poverty and energy poverty originated separately, they
began to influence each other and converge in the 2010s.Bouzarovski \&
Petrova (2015) further embraced that both fuel and energy poverty can be
viewed under a conceptual umbrella, describing them as ``a set of
domestic energy circumstances that do not allow for participating in the
lifestyles, customs, and activities that define membership of society''.
Hence in this article, we adopt the term ``energy poverty'' herein not
only to align with mainstream usage in Taiwan, but also, to account for
the multi-dimensional nature of energy deprivation.

\hypertarget{energy-poverty-landscape}{%
\section{Energy Poverty landscape}\label{energy-poverty-landscape}}

\hypertarget{global-north}{%
\subsection{Global North}\label{global-north}}

In the Global North, energy poverty is primarily framed around issues of
affordability (Chan \& Delina, 2023), as colder climate significantly
contribute to high energy consumption for heating, making the cost of
energy a critical factor. Affordability in this context refers to the
relatively high financial burden imposed by energy costs, which often
leads households to self-provision, choose between heating and other
essentials, or endure inadequate thermal comfort. While income plays a
crucial role, energy poverty is distinguished from income poverty
(Phimister et al., 2015) because it involves a complicated interplay
between income, energy efficiency, and housing characteristics.

The form of energy poverty that households in the Global North are
experiencing is characterized by high energy costs relative to household
income, poor energy efficiency in housing, and socioeconomic
vulnerabilities of affected populations. Energy prices in the Global
North are relatively high, especially during winter months when fuel
demand peaks or in the context of unstable geopolitical situations (Guan
et al., 2023). The cost of heating fuels significantly impacts household
budgets, causing households to spend a disproportionate amount of their
income on basic needs, particularly low-income households (Liu et al.,
2023; Zapata-Webborn et al., 2024). A substantial portion of housing
stock is aged and inadequately insulated, leading to increased energy
demand for heating (Sánchez-Guevara et al., 2019). The lack of
sufficient insulation, outdated heating systems, and
low-energy-efficient appliances contribute to considerable heat loss,
making it increasingly challenging to maintain a comfortable indoor
environment (Middlemiss, 2022). Socio-economically disadvantaged groups,
such as low-income households and elderly people, often struggle to
balance spending on energy with other necessities, resulting in
inadequate thermal comfort and poor quality of life, thereby
exacerbating their marginalized status and increasing their
susceptibility to disease and social exclusion(Brandon \& Bagnall,
2024). The neoliberalization of energy and housing markets, along with
deepening relative inequality, has significantly contributed to the
emergence of new challenges related to energy poverty (Robinson, 2019).
Such inequalities create a vicious cycle where historical inequities,
market forces, and social and political exclusion reinforce spatial
inequalities, leading to disproportionately high energy burdens for
marginalized communities (Bouzarovski \& Simcock, 2017).

To address the issue of energy poverty, various policies have been
implemented in the Global North. The Hungarian government adopted
policies to regulate energy prices to mitigate the impact of price
fluctuations on vulnerable households, including introducing price
support schemes in the mid-2000s that subsidized domestic natural gas
consumers, effectively reducing energy expenditures for up to 80\% of
households at its peak in 2006 (Bouzarovski et al., 2016). Social
tariffs and fuel allowances can provide direct financial relief and
offset the cost of energy for eligible households (Herrejón et al.,
2023). Given the significant contribution of energy-inefficient housing
to energy poverty and climate change, policies that promote energy
efficiency retrofits, encompassing measures like insulation upgrades,
installation of energy-efficient windows, and replacement of outdated
heating systems, have gained increasing attention (Middlemiss, 2022).
Financial support schemes like green loans can incentivize homeowners,
particularly those in poorly insulated houses, to undertake these
improvements (Herrejón et al., 2023). The underlying socio-economic
factors significantly contribute to energy poverty; therefore, policies
regulating the housing market, especially the rental market, have been
deployed to ensure access to affordable and energy-efficient housing
(Middlemiss, 2022). These policies involve setting minimum energy
efficiency standards for rental properties, providing financial
assistance for energy-efficient upgrades to rental units, and
implementing rent control measures to prevent excessive housing costs,
especially in areas with high energy costs.

Recent research on energy poverty in the Global North has expanded to
include a deeper understanding of vulnerability, intersectionality, and
the lived experiences of those affected (Middlemiss, 2022). Overlapping
social categories such as age, gender, race, and socioeconomic status
intersect to influence an individual's experience of energy poverty
(Bouzarovski \& Bouzarovski, 2018). Qualitative methods are gaining more
attention to understand the day-to-day struggles and coping strategies
of energy-poor households (Herrejón et al., 2023). However, much of the
existing literature is dominated by UK and European studies, partly
because the term ``fuel poverty'' originated there, limiting the
understanding of how energy poverty manifests in different regions and
local contexts across the Global North; such a focus may neglect the
need for cooling in hotter climates, which is increasingly important due
to rising global temperatures, and more frequent heatwaves (Bouzarovski
\& Bouzarovski, 2018). Regions in southern Europe, parts of the United
States, and Australia face significant challenges related to extreme
heat, which can have severe health impacts during heatwaves.

The evolving understanding of energy poverty in the Global North
highlights the importance of recognizing the intersectionality of
vulnerability factors such as age, gender, race, and socioeconomic
status. While qualitative methods are increasingly utilized to capture
the lived experiences of energy-poor households, the predominance of UK
and European studies limits our comprehension of energy poverty across
diverse regions. As climate change exacerbates the energy need, it is
crucial to expand research efforts to include areas facing extreme heat,
such as southern Europe, parts of the United States, and Australia.
Addressing these gaps in knowledge will enable policymakers to develop
targeted strategies that consider the unique challenges faced by
different communities, ultimately leading to more effective
interventions to alleviate energy poverty and improve the quality of
life for affected individuals.

\hypertarget{global-south}{%
\subsection{Global South}\label{global-south}}

In the Global South, energy poverty is primarily framed around issues of
accessibility (Herrejón et al., 2023). The relatively underdeveloped
energy infrastructure significantly contributes to the challenge of
accessing reliable and clean energy sources for essential needs like
lighting, cooking, and cooling. Accessibility in this context refers to
the difficulty in securing these energy sources, often leading
households to rely on traditional biomass, experience inadequate
cooling, or face intermittent electricity supply.

The form of energy poverty experienced by households in the Global South
is characterized by underdeveloped energy infrastructure, dependence on
traditional biomass fuel, and socioeconomic disparities, including
urban-rural divides (Chan \& Delina, 2023). Many regions in the Global
South suffer from inadequate energy infrastructure, resulting in
unreliable electricity supply and frequent outages (Hafner \&
Tagliapietra, 2020). Access to electricity or modern fuel is
inconsistent, with rural and remote areas being particularly on the
periphery due to the lack of grid connection (Hostettler, 2015).
Consequently, households heavily rely on traditional biomass fuel like
wood, coal, and even dung for cooking and heating, which are relatively
inefficient and contribute to increased CO emissions and indoor air
pollution (Kumar et al., 2019). The use of such fuels has significant
health threats, especially for women and children who are responsible
for cooking (Jessel et al., 2019). Reliance on traditional biomass fuels
leads to severe respiratory issues and other health problems due to
exposure to smoke and pollutants. Moreover, the lack of reliable energy
access jeopardize the provision of essential services, such as
healthcare and education, both of which require stable electricity to
function effectively. The labor-intensive nature of collecting
traditional fuels also deprives children, particularly girls, of the
opportunity to receive a proper education, as they often assist in
gathering fuel (Longe, 2021). Such conditions create a vicious cycle
where inadequate infrastructure, economic disparities, and social
exclusion reinforce each other, leading to persistent energy poverty.
This cycle perpetuates inequalities, limiting opportunities for economic
and social development and trapping marginalized communities in a state
of deprivation.

To address energy poverty in the Global South, interventions aimed at
improving energy accessibility are being implemented (Bouzarovski \&
Petrova, 2015; Chan \& Delina, 2023). These include, but are not limited
to, infrastructure development and modernization, rural electrification
programs such as the transition to renewable energy, and economic
policies like subsidies and microfinance. India's ``Power for All by
2012'' initiative was an ambitious program aimed at ensuring a reliable
and quality power supply to all citizens by the year 2012. Although the
program did not fully meet its target within the stipulated time, it
laid significant groundwork for subsequent rural electrification efforts
and overall improvements in the power sector (Niez, 2010). In countries
like Kenya, rural electrification efforts have been significantly
boosted through comprehensive strategies involving grid extension and
subsidies. These efforts have not only increased electrification rates
but have also profoundly impacted the socio-economic development of
rural areas (Njiru \& Letema, 2018). Subsidies, such as the Free Basic
Electricity (FBE) program by the South African government, provide 50
kWh of free electricity monthly to indigent households connected to the
grid (Monyei et al., 2018). This program can alleviate energy poverty by
ensuring that vulnerable citizens have access to essential electricity
for their basic needs. However, the government also acknowledges that
the program does not reach all indigent households, especially the most
vulnerable, as they are not connected to the grid (Longe, 2021).

Recent discourse around energy poverty in the Global South heavily
emphasizes the importance of ``energy justice,'' which involves the fair
distribution of energy benefits and burdens, recognizes the needs of all
stakeholders, and establishes a fair decision-making process
(Samarakoon, 2019). The emerging focus on energy justice challenges the
prevailing narrative centered on accessibility issues, shifting
attention to the sufficiency of energy supply and the quality of energy
services required for improved well-being (Monyei et al., 2018). As the
understanding of energy poverty deepens and becomes more holistic, there
is a growing emphasis on centering the lived experiences of energy-poor
communities and acknowledging that people's energy needs are diverse and
context-specific (Mastrucci et al., 2019). Further enriching this
landscape is the increasing recognition of the gender dimension, as
women and girls are more likely to be affected and marginalized by
energy poverty (Kumar et al., 2019; Longe, 2021). Recognizing these
gendered impacts is crucial for developing effective and equitable
interventions to address energy poverty.

In conclusion, addressing energy poverty in the Global South requires
diverse approaches that acknowledge the complexities of accessibility,
infrastructure, and socio-economic disparities. The reliance on
traditional biomass fuels and the challenges posed by underdeveloped
energy infrastructure not only limited the access to reliable energy
sources but also have significant health and social implications for
affected communities. As interventions such as infrastructure
development, rural electrification, and economic policies are
implemented, it is essential to prioritize energy justice, ensuring that
the distribution of energy benefits and burdens is equitable and
inclusive. Furthermore, centering the lived experiences of energy-poor
communities and recognizing the specific needs of women and girls will
enhance the effectiveness of these interventions.

\hypertarget{measuring-energy-poverty}{%
\section{Measuring Energy Poverty}\label{measuring-energy-poverty}}

To properly address the issue of energy poverty, measuring is vital yet
difficult to do because of the multifaceted nature(Thomson et al.,
2017), it is not merely a lack of access to energy but is deeply
intertwined with broader socio-economic conditions such as household
income, education levels, and demographic characteristics. Several
energy poverty indicators are developed to capture the multifaceted
nature, yet, they have their own strengths and limitations. Tailored to
Taiwan's energy poverty, only expenditure-based methods are reviewed in
this article.

Expenditure-based approaches are the most widely used method for
quantifying energy poverty, which assess the ratio of a household's
income to its energy expenditure to determine whether the household
falls into energy poverty. We hereby examine five main indicators of
energy poverty: Ten Percent Rule (TPR), Low-income High Cost (LIHC),
Minimum Income Standard (MIS), Double Median (2M), and Half Median
(M/2).

%%%%%%% tbl:ep-calculation %%%%%%%%
\begin{table}[!htb]
    \centering
    \resizebox{\columnwidth}{!}{%
    \begin{tabular}{@{}ll@{}}
    \toprule
    Indicator              & Calculation                                                \\ \midrule
    Ten Percent Rule (TPR) & (energy expenditure / disposable income) \textgreater 10\% \\
    Double Median (2M)     & energy expenditure ≥  (2 * median energy expenditure)      \\
    Half Median (M/2)      & energy expenditure ≤ (median energy expenditure / 2)       \\
    Low-Income High Cost (LIHC)   & disposable income ≤ (median disposable income * 0.6) \& energy expenditure ≥ median energy expenditure \\
    Minimum Income Standard (MIS) & energy expenditure ≥ (disposable income - housing cost - minimum living costs)                         \\ \bottomrule
    \end{tabular}%
    }
    \end{table}
%%%%%%% END %%%%%%%%

\begin{landscape}
    \begin{table}[]
    \centering
    \resizebox{\columnwidth}{!}{%
    \begin{tabular}{@{}ll@{}}
    \toprule
    Indicator              & Calculation                                           \\ \midrule
    Ten Percent Rule (TPR) & (energy expenditure / disposable income) > 10\%        \\
    Double Median (2M)     & energy expenditure ≥  (2 * median energy expenditure) \\
    Half Median (M/2)      & energy expenditure ≤ (median energy expenditure / 2)  \\
    Low-Income High Cost (LIHC)   & disposable income ≤ (median disposable income * 0.6) \& energy expenditure ≥ median energy expenditure \\
    Minimum Income Standard (MIS) & energy expenditure ≥ (disposable income - housing cost - minimum living costs)                        \\ \bottomrule
    \end{tabular}%
    }
    \end{table}
    \end{landscape}

The \protect\hyperlink{tbl:ep-calculation}{table},

The TPR was proposed by Boardman in 1991 on fuel poverty in the UK,
defining it as ``the inability to afford warmth''. To determine the
``affordable'' energy cost, Boardman proposed the ten percent threshold,
suggesting that households spending more than this proportion are
experiencing ``undue financial hardship'', hence requiring
assistance.Boardman (1991) explored the multifaceted issue through
several aspects, identifying that inadequate housing condition, low
incomes, and high energy cost are the primary drivers. Attaining
adequate warmth is contingent upon the energy efficiency of the home and
heating system, not just income levels, meaning even households with
income above the poverty line could experience fuel poverty if living in
energy-inefficient housing with high heating cost. Solely focusing on
income-based solutions may appear beneficial in the short term, but it
would be costly and insufficient to address the underlying problem of
energy-inefficient housing in the long run.

On the surface, the TPR appears to be an absolute metric and maybe
somewhat arbitrary; however, it is actually partly based on the relative
concept of twice the median(Boardman, 2012), which was introduced
byIsherwood \& Hancock (1979) in response to the rapid increase in
household energy bills due to the energy crisis in the 1970s. According
toBoardman (1991), the TPR reflects the actual average share of energy
spending among the 30 percent poorest households in Great Britain, which
is around twice the median share of energy spending for all households
at that time. While the usage of double the median as an indicator can
trace back all the way toIsherwood \& Hancock (1979), but it is
difficult to come by the actual content to analyze why this is a good
threshold.Heindl (2013) examined~the definition of fuel poverty lines
and measurement techniques, including options such as double the median
and mean household expenditure on energy, as well as double the median
and mean share of household expenditure on energy, highlighting that
different lines significantly affect the identification of fuel-poor
households. Upon discussing energy poverty threshold, median indicators
precede mean indicators as the energy expenditure are usually
right-skewed, the median would be less less affected by extreme values
and outliers, providing a more accurate reflection of typical household
expenditures, thus ensuring that the measurement better captures the
financial burden faced by lower-income households.

Whilst most British researchers generally accepted double the median
share of energy expenditure as a guiding consideration,Schuessler (2014)
argues that the orthodox understanding of the TPR would be problematic
as the twice the median indicator has serious methodological problems;
hence, the interpretation and justification should be based on energy
expenditure at the lowest income strata as this energy poverty threshold
helps identify households in need. Although the double the median share
indicator is superior than the double the mean share indicator, the
double the median indicator inherently have some awkward properties that
makes it unsuitable to measure energy poverty. Because the nature of
median is based on the order of the data, instead of the actual numbers,
therefore, adding a constant to an otherwise unchanged distribution will
decrease the number of households exceeding the double median in
right-skewed distributions. In other words, ceteris paribus, adding a
fixed energy cost to all households could reduce the total amount
energy-poor households. The counterintuitive phenomenon not only
showcase the fundamental flaws of the 2M indicator, it also violatesSen
(1976)'s monotonicity requirement for poverty measures, which states
that that poverty should increase when the well-being of an already poor
household deteriorates. This violation of the monotonicity requirement
highlights the inadequacy of the 2M indicator for accurately measuring
energy poverty, as it does not respond proportionally to worsening
conditions among the poorest households.

There are several limitations when directly applying the TPR threshold
to other countries, because the indicator was UK-specific as it was
derived from UK 1988 data(Herrero, 2017). Without thoroughly considering
local context----such as climate conditions, housing, and cultural
norms----the metric may not be able to reflect the vulnerable group.
Besides the varying geographical context, the potential problem is the
differences between required energy cost and actual energy expenditure.
The TPR was originally developed based on actual spending on fuel;
however, utilizing actual energy expenditure may underestimate the
prevalence of energy poverty as households struggling to afford adequate
warmth may ration their energy usage, therefore, unable to be detected
by the TPR(Boardman, 2012; Herrero, 2017; Riva et al., 2023). To address
this issue, the focus has transitioned towards using required energy
cost, which aims to estimate the cost of energy services a household
needs to achieve a defined standard of warmth. However, the required
energy cost, or modeled energy bill, heavily relies on UK's detailed
housing survey, which may not be viable for other nations. Therefore,
the M/2 indicator was developed to capture the hidden energy poverty
situation, where the self rationing behavior exists. The M/2 indicator
identifies households with equal or less then half of the median energy
expenditure of all households as hidden energy-poor households.

The Low-Income High Cost (LIHC) indicator was developed byHills (2012),
classifies a household as energy poor if they have required fuel costs
that are above the contemporary median level for all households; and
were they to spend that amount on fuel, they would be left with a
residual income below the official poverty line. The official poverty
line is calculated as 60\% of median income after housing costs. As the
size of households matter to the energy usage, the methodology allows
for differences in household size and composition using a set of
equivalization factors. The invention of the LIHC indicator was to
remedy the flaws stemming from the adoption of a fixed threshold, making
it susceptible to various inaccuracies and misrepresentations of the
fuel poverty landscape. According toHills (2012), the traditional TPR
are flawed due to the arbitrary nature, hypersensitivity to price
fluctuations, and the inclusion of non-poor households. The initial
justification for setting the threshold at 10\%, which was twice the
median household fuel expenditure, lacks a robust rationale. The TPR
being a fixed threshold makes it highly susceptible to changes in energy
price, creating a misleading picture of energy poverty landscape. And
since the TPR solely focuses on the proportion of income spent on
energy, it can misclassify households with moderate or high incomes as
fuel poor, distorting the actual scale of energy poverty and can hence
misdirect policy interventions.

Unlike the TPR only focus on income and energy expenditure, the LIHC is
calculated after deducting the housing cost, which may be more accurate
for several reasons. As households cannot allocate income designated for
housing cost towards energy bills, deducting housing cost represents a
more accurate picture of the disposable income available for energy
expenses. Another advantage of calculating energy poverty after the
deduction of housing cost is the ability to eliminate geographic
variability in housing cost. Even if the housing condition remains
constant, the price can vary considerably based on location. Without
deducting housing costs, households may appear to be lifted out of
energy poverty simply because of the increasing rent. Overall, deducting
housing costs in the LIHC calculation is crucial for providing a more
accurate and realistic assessment of household disposable income. Even
though LIHC as an indicator has many advantages when capturing the
energy-poor households, it is not free of criticisms. Due to the doubly
relative nature of LIHC, the indicator will remain relatively stable
even when energy price fluctuated. Even if the energy price rise
significantly, the median energy cost might not change drastically,
especially if the energy consumption patterns across income level remain
relatively stable. The insensitivity can mask the impact of rising
energy prices on low-income households and may lead to underestimate of
the true extent of energy poverty.

Moore (2012)'s Minimum Income Standard (MIS) approach considers~a
household to be energy poor~if, after meeting its energy costs, it does
not have enough income left to meet the~minimum income standard~for an
acceptable standard of living. The MIS is calculated by establishing a
basket of goods that a household need for an acceptable standard of
living. Unlike relative measurement like LIHC, which rely on
median-based threshold, MIS is more need-based oriented, ensuring the
indicator captures households whose remaining income is inadequate for a
decent quality of life. MIS reflects the actual costs faced by
households, including rising energy prices, housing conditions, and
changes in essential goods. This ensures that the measure is more
responsive to real-world conditions, such as inflation or fluctuations
in energy prices, compared to purely relative indicator. However, the
MIS is data-intensive and gathering and maintaining the data can be more
costly. Compare to measures like LIHC, the MIS requires detailed
information about household spending needs, energy consumption patterns,
and regional cost-of-living variations. This limitation makes it harder
to implement at scale, especially in contexts where such detailed data
may not be readily available.

Expenditure-based methods provide a direct assessment of the financial
burden of energy costs on households and the data required, such as
household income and energy expenditure, are relatively straightforward
to collect through surveys, which several governments have been
executing for years. However, the limitations are also prominent as they
may not be able to capture all dimension of energy poverty, such as the
adequacy of energy services or the energy efficiency of homes. Most of
the countries, except for UK, use actual energy expenditure instead of
required spending, which is problematic since it fails to capture the
self-restricting behavior, overlooking the most vulnerable group(Moore,
2012). On the other hand, theoretical cost can be more accurate but
requires detailed data on housing energy efficiency, like UK's housing
condition surveys, which is not available for most countries.

\hypertarget{energy-poverty-in-taiwan}{%
\section{Energy Poverty in Taiwan}\label{energy-poverty-in-taiwan}}

Considered a developed economy, Taiwan shows distinctive energy
transition path compare to other free market economies(Chan \& Delina,
2023) due to the developmental state legacy(Pien et al., 2023). Path
dependent on brown economy regime, Taiwan's economy is~characterized by
high carbon emissions, cheap energy prices, and the dominance of
energy-intensive industries, which makes it challenging to transition
towards a more sustainable, low-carbon energy system(Chou \& Liou, 2023;
Kemp-Benedict, 2014). Taiwan has undergone rapid industrialization and
urbanization in recent decades, leading to a significant increase in
energy demand(Su, 2019, 2020).

Building upon the developmental state heritage, Taiwan's power supply is
dominated by the state-owned Taiwan Power Company (Taipower), which has
a monopoly on electricity generation, transmission, and distribution.
Taiwan has achieved a very high rate of electricity access, however, it
does not necessarily mean there are no issues with energy affordability
and deprivation, especially for economically disadvantaged households.
The energy billing system adopts progressive charging pricing method,
charging higher rates per unit of energy as consumption increases,
further exacerbating the difficulties experienced by disadvantaged
groups. Lacking of access to energy efficient appliances, low-income
households typically spend a higher proportion of their income on energy
and have less flexibility to adjust their consumption patterns due to
progressive charging rate.

Taiwan is undergoing a significant energy transition, moving away from
nuclear power towards greater reliance on natural gas and renewable
energy sources. The Taiwan government has released the ambitious
Net-Zero transition agenda by 2050, as nuclear power is phased out, LNG
imports have increased(National Development Council, 2024). While
natural gas is cleaner than coal, it is more expensive and sensitive to
price fluctuation, which can impact household energy costs even with
Taipower trying to mitigate the price. The energy transition can
intensity energy poverty by increasing costs and reducing supply
reliability. To reflect the price increase due to unstable geopolitical
situation, residential energy price has gone up. Households already
struggling with energy costs may find it even harder to afford their
needs during energy transition.

The energy transition in Taiwan is likely to have significant
socioeconomic implications, including potential job losses in
traditional energy sectors and the need for retraining and support for
affected workers and communities(Chien, 2022). Understanding the
distributional impacts of these changes on vulnerable populations is
crucial for designing inclusive energy policies. Taiwan is experiencing
a rapidly aging population and changing household dynamics, with more
single-person and elderly households(Cheng, 2023). This demographic
trend emphasize the relevance of further investigate the household
vulnerability aspect. Previous research have discussed the crucial role
of energy poverty in deciding household energy demand(Su, 2020),

Taiwan's unique combination of historical, political, and socio-economic
factors provides a rich context for examining the complexities of energy
poverty. The interplay between the developmental state legacy, the
ongoing energy transition, and specific household characteristics offers
valuable insights for both policymakers and researchers. By exploring
these aspects in detail, this research aims to contribute to a more
comprehensive understanding of energy poverty and inform the development
of more effective and equitable energy policies in Taiwan and beyond.

Energy poverty in Taiwan is gaining increasing attention in academic
field, however, the government has not formally recognize energy poverty
as a problem distinct from general poverty. The current structure of the
electricity market and socio-economic factors create unique
vulnerabilities that need to be thoroughly understood to develop
effective and equitable policies. To evaluate the vulnerability
characteristics of energy poverty, we proposed hypotheses focusing on
Taiwan's household head's characteristics and household structure.

\hypertarget{household-heads-characteristics}{%
\subsection{Household Head's
Characteristics}\label{household-heads-characteristics}}

Building on previous research on the relationship between household
head's and the risk of energy poverty, gender plays an important role in
affecting the vulnerability. Energy poverty is a structural problem,
when it comes to affordability issue, it is inevitable to face the
gendered income inequalities. Female-headed households have poverty rate
s almost twice as high as single/male-headed household in the US(Sharma,
2023); this is largely due to the persistent gender wage gap. In
societies with higher gender equality, the gender of the household head
may be less indicative of energy poverty risk. However, in a more
unequal settings, it can serve as a significant indicator of household
dynamics and resource allocation, influencing vulnerability to energy
poverty. Therefore, understanding the role of household head
characteristics, particularly gender, is crucial in assessing and
addressing energy poverty. In Taiwan, income disparities between
male-headed and female-headed households remain significant. According
to the 2021 Report on Family Income and Expenditure, male-headed
households have a higher average disposable income (1,185,000 NTD)
compared to female-headed households (880,000 NTD) (DGBAS, 2022). While
per capita disposable income shows no significant difference when
adjusting for household size, energy poverty calculations are
traditionally based on a per-household basis. This suggests that
female-headed households may be more vulnerable to energy poverty.

\begin{quote}
Hypothesis 1: Female-headed households in Taiwan are more likely to
experience energy poverty compared to male-headed households.
\end{quote}

\hypertarget{family-structure}{%
\subsection{Family Structure}\label{family-structure}}

Taiwan is facing an aging population and declining birth rate, leading
to a decreasing size of households, from 3.4 in 2001 to 2.6 in 2021.
Households of varying sizes may have different patterns of electricity
usage and energy consumption, as the economies of scale play a crucial
role in determining households' energy consumption, larger households
tend to be more energy-efficient per person due to the fixed-cost of
energy services being distributed among to more individuals. This
phenomenon is well-documented in the literature, and aligned with the
findings in Taiwan(Huang, 2015; Su, 2020). Policy recommendations hence
are more incline to target at smaller households as they consume more
electricity per person in comparison to other family composition type
for electricity conservation. While economies of scale significantly
impact household energy consumption patterns, it is crucial to
distinguish between energy consumption and energy poverty in Taiwan.
Policy recommendations targeted at smaller households for electricity
conservation may not adequately address the issue of energy poverty,
which is closely related to affordability issue. From family structure
perspective, energy poverty is not solely determined by household size
but it is also influenced by the composition. Single individuals or
larger households may face greater risk of energy poverty, even with
economies of scale to offset some portion of energy use.

\begin{quote}
Hypothesis 2.1: The risk of a household falling into energy poverty in
Taiwan shows a non-linear relationship with household size, where the
risk decreases as household size increases up to a certain point, after
which the risk begins to rise again.~
\end{quote}

\begin{quote}
Hypothesis 2.2: Certain household compositions are associated with a
higher risk of energy poverty in Taiwan. Specifically, single-person
households and skip-generation households (grandparents with
grandchildren) are more likely to experience energy poverty compared to
nuclear and extended families, due to factors like lower total income
and less benefit from economies of scale.
\end{quote}

\begin{quote}
Hypothesis 2.3: Household composition moderates the non-linear
relationship between household size and energy poverty risk. The
decrease in energy poverty risk with increasing household size varies by
composition, being more significant in nuclear families than in
single-parent or skip-generation households.
\end{quote}

\hypertarget{sec:methodology}{%
\chapter{Methodology}\label{sec:methodology}}

\hypertarget{data}{%
\section{Data}\label{data}}

The data source used in this research is the 2021 Survey of Family
Income and Expenditure (SFIE) collected by the Directorate General of
Budget, Accounting, and Statistics(DGBAS, 2022). The COVID-19 outbreak
in 2021 significantly affected Taiwan's economy and households. The
pandemic may have exacerbated energy poverty due to factors like
increased unemployment, reduced incomes, or more time spent at home
leading to higher energy consumption. Studying this year can provide
insights into how such external shocks influence energy poverty and have
better understanding of the vulnerable groups.

The 2021 survey includes 16,528 households with 46,894 members from 20
administrative regions in Taiwan. The cross section covered six urban
regions (New Taipei City, Taipei City, Taoyuan City, Taichung City,
Tainan City, and Kaohsiung City) and other fourteen regions in Taiwan
(Yilan County, Hsinchu County, Miaoli County, Changhua County, Nantou
County, Yunlin County, Chiayi County, Pingtung County, Taitung County,
Hualien County, Penghu County, Keelung City, Hsinchu City, Chiayi City,
and Kinmen County). The sample size within these regions range from 200
in Penghu County to 2,500 in Taipei City. The survey employed a
stratified two-stage sampling method with counties and cities as
subpopulations, where regional sample counts varied by population size
to ensure proportional representation. The Ts'un and Li, which are basic
administrative units, serve as the primary sampling units, and
households within these units are the secondary sampling units. The
intra-cluster correlation caused dependence issue, which violate
single-leveled models' independence assumption. If the dependence
structured is ignored, it will result in the underestimation of standard
errors, inflated Type I error rates, and misleading inferences. The
standard errors will be underestimated because the similarity among
observations within clusters is ignored. In reality, households within
the same region may share unobserved characteristics, such as energy
infrastructure and region specific social welfare project, leading to
correlated errors. Because standard errors are underestimated, the
likelihood of incorrectly rejecting the null hypothesis increases. This
leads to false positives, where the model indicates that an effect
exists when it actually does not; The overall inferences drawn from the
model can be misleading(Hox et al., 2017).

The SFIE dataset exhibits a two-level hierarchical and aggregated
structure, with households nested within broader regional categories
such as counties and cities. This hierarchical organization allows for
the analysis of both within-region and between-region variations,
facilitating a comprehensive understanding of the factors influencing
energy poverty. Before 2007, the survey contains the administrative area
code and stage information. Given privacy concerns, the precise
locations of households are not disclosed since. To address this, we
examined workplace data (variables from b15\_1 to b15\_20) included in
the survey and cross-referencing it with governmental reports. The
method may appear somewhat basic, yet it has proven to be effective and
reliable for our purposes, as evidenced by similar approaches used in
previous studies, such asK.-M. Chen \& Wang (2015) . Since the
government report list out sample counts for each region, we can further
confirm the region data based on the aggregated structure. This approach
ensured respondent anonymity while enabling the incorporation of
regional variables into our analysis.

At the household level, the SFIE dataset includes a comprehensive array
of variables capturing detailed information on socio-economic status,
demographic characteristics, and expenditure patterns. Key income
variables include total household income from all sources and per capita
income, providing measures of economic status. Expenditure variables
capture total household spending across categories such as food,
housing, utilities, education, and healthcare, with detailed energy
expenditure statistics, which is critical for studying energy poverty.
Demographic variables such as household size, age, and composition, shed
lights on the dynamic between energy poverty and family structure.

\hypertarget{dependent-variable}{%
\section{Dependent Variable}\label{dependent-variable}}

Traditionally, the manifestation of energy poverty in the Global North
and Global South is divided by the affordability-accessibility
dichotomy, focusing on the lack of adequate heating versus the lack of
access to clean energy. As the field developed, the term energy poverty
has emerged as an concept umbrella encompassing multiple interconnected
issues: access to energy sources, infrastructure development,
affordability of energy services, energy efficiency, equity in
distribution, as well as the impacts on human well-being and the
environment(Herrejón et al., 2023). This article acknowledges the
critics about the potential threats of oversimplifying energy poverty as
affordability or accessibility issue and the trend towards using
consensus approach to define energy(Guevara et al., 2022), however, as
the purpose of this article is to examine the risk of vulnerable group
and their affordability under Taiwan's ambitious 2050 Net-Zero
transition agenda, using uni-dimensional indicator is still valid.

In this article, the dependent variable is the energy poverty status of
households, measured thorugh the Low-Income High Cost (LIHC) indicator
as proposed byHills (2012) . The LIHC approach is particularly suitable
for capturing the affordability aspect of energy poverty in Taiwan, as
it identifies households that have both low income and high energy costs
relative to the median. The total household income is represented by the
variable itm400, which includes all forms of receipts. To determine the
residual income available for energy and other living expenses,
non-consumption expenditures (itm600) are deducted from the total
receipts. Following the LIHC methodology, housing costs are deducted
from the disposable income to evaluate the actual residual income
households can allocate for other living expenses. The housing costs are
computed by summing the actual rentals for housing (itm1041), imputed
rentals of owner-occupiers (itm1042), and imputed rentals of issued and
leased (itm1043). Households with residual incomes below poverty
threshold, 60\% of the median residual income after housing costs, are
classified as low-income households. Next, the study identifies
households with high energy costs by calculating the median energy
expenditure across all households. Energy expenditures are derived from
the SFIE data, which includes detailed information on household spending
on utilities such as electricity (itm1047), gas (itm1048), and liquid
and solid fuels (itm1049). Households energy expenditures exceed the
median energy cost are considered to have high energy costs. A household
therefore be classified as energy-poor if the residual income is below
the poverty line, and the energy expenditure is above the median energy
cost. If a household does not meet both conditions, it is classified as
not energy poor.

To control for the effects of household size and composition on energy
spending, the study employs a modified OECD equivalence scale to adjust
income and energy expenditure thresholds. This equivalence scale
accounts for economies of scale in household consumption, allowing for a
more accurate comparison between households of different sizes and
compositions. The scale assigns a weight of 1.0 to the first adult
(household head), 0.5 to each additional adult member aged 15 years or
older, and 0.3 to each child under the age of 15 years (Hagenaars et
al., 1994). The total equivalence scale for a household is calculated
using the following formula:

\hypertarget{independent-variables}{%
\section{Independent Variables}\label{independent-variables}}

This article examines the risk of households falling into energy poverty
from two aspects: household head's characteristics and family structure.
As noted in the discussion of Taiwan, household heads' gender may have
significant implication of certain vulnerability. The attribute of the
gender data is stored in column a6, with the value 1 refers to male and
female as value 2, allowing for categorical analysis in the model.

Household size data is stored in a7 as a continuous variable, ranging
from 1 to 11. To capture the non-linear relationship, this article
created a squared term of household size to capture both the initial
decrease in risk as household size increase and the subsequent increase
in risk beyond a certain population. Family structure and composition
data is stored in column a18, which includes detailed coding of the
gender and generation of the household head. To analyze the effect of
family composition on the energy poverty, this study recodes the
variable into a simplified set of composition categories: one person,
married couple, single parent, nuclear family, ancestors and
descendants, extended family, and others.

\hypertarget{control-variables}{%
\section{Control Variables}\label{control-variables}}

In addition to household head's gender, household size, and family
composition, this study controls several variables from household head
characteristics and household structure factors that may potentially
influence the risk of energy poverty. The primary factor that have been
proven to affect households' electricity consumption is education
attainment. Higher level of education often correlate with better job
opportunities and higher income. Household heads with higher education
may have more knowledge about energy-efficient technologies and
practices, which significantly influence the energy usage.

\hypertarget{model}{%
\section{Model}\label{model}}

The dependent variable in this study is the LIHC indicator, a binary
measure indicating whether a household is in energy poverty (1) or not
(0). The LIHC status is determined based on a household's energy
expenditure relative to its income. Given the hierarchical nature of the
data, where households are nested within counties, a Generalized Linear
Mixed Model (GLMM) is adopted. This model allows for the inclusion of
both fixed effects, household-level and county-level predictors, and
random effects, to which account for the variability across counties,
making it suitable for handling the dependencies introduced by the
two-stage stratified sampling method. The model is described as follow:

\hypertarget{robustness-checks}{%
\section{Robustness Checks}\label{robustness-checks}}

To ensure the validity and reliability of our findings derived from the
model, we will conduct several robustness checks from both model
assumption examination and variables calculation aspects. In regard of
the data sampling method of SFIE dataset, the multilevel structure is
inherent, therefore, treating countyname as fixed effects instead of
random effects will not be performed. Fitting a single level model with
countyname as fixed effect requires separate parameter estimates for
each of the 20 counties, although it may help with understanding each
county's energy poverty distribution individually, it would not capture
the inherent hierarchical nature of the data and would lack
generalizability. Since the focus of this analysis is on broader trends
in energy poverty, focusing on household's condition, rather than on
county-specific insights, a random effects structure is more
appropriate. Instead, we will compare the generalized linear mixed model
and hierarchical generalized linear model approaches to determine the
best fit for our data. One of the key different assumption is the
normality of random effects. GLMM assumes that the random effects are
normally distributed, on the other hand, HGLM allows for more
flexibility with distributional assumptions in the random effects, which
could better capture county-level variations if the normality assumption
proves inadequate.

To check the robustness of the result, we will also try square root
scale equivalized method for determining energy poverty. In this study,
OECD modified equivalence scale is adopted due to its wide acceptance in
the field. Square root scale, unlike OECD modified scale assigns
different weights to adults and children, treats all household members
equally in its adjustment factor. Both the OECD-M and square root scales
reflect the concept of economies of scale, where larger households are
more efficient in their energy use per capita due to shared facilities
and energy consumption patterns. However, the square root scale
simplifies this by treating all household members equally, which may
suit analyses that emphasize size over composition. Comparing the
different scale not only will help ensuing the validity of the model
results, but also provide insights into whether composition or size
matters more when it comes to energy poverty, which can be further
considered during policy intervention.

\hypertarget{sec:results}{%
\chapter{Results}\label{sec:results}}

\hypertarget{sec:conclusion}{%
\chapter{Conclusion}\label{sec:conclusion}}

\hypertarget{references}{%
\chapter*{References}\label{references}}
\addcontentsline{toc}{chapter}{References}

\hypertarget{refs}{}
\begin{CSLReferences}{1}{0}
\leavevmode\vadjust pre{\hypertarget{ref-Akram2022CausalityBA}{}}%
Akram, V. (2022). Causality between access to electricity and education:
Evidence from BRICS countries. \emph{Energy RESEARCH LETTERS}.

\leavevmode\vadjust pre{\hypertarget{ref-Aramillo2007ComparativeLA}{}}%
Aramillo, P. J., Ichael, W. M., Riffin, G., \& Atthews, H. S. C. M.
(2007). Comparative life-cycle air emissions of coal, domestic natural
gas, LNG, and SNG for electricity generation. \emph{Environmental
Science \& Technology}, \emph{41}, 6290--6296.

\leavevmode\vadjust pre{\hypertarget{ref-Baum2022EnergyAA}{}}%
Baum, F., McGreevy, M., MacDougall, C. M., \& Henley, M. (2022). Energy
as a social and commercial determinant of health: A qualitative study of
australian policy. \emph{International Journal of Health Policy and
Management}, \emph{12}.

\leavevmode\vadjust pre{\hypertarget{ref-Boardman1991FuelP}{}}%
Boardman, B. (1991). \emph{Fuel poverty : From cold homes to affordable
warmth}. \url{https://api.semanticscholar.org/CorpusID:142746739}

\leavevmode\vadjust pre{\hypertarget{ref-Boardman2012FuelPS}{}}%
Boardman, B. (2012). Fuel poverty synthesis: Lessons learnt, actions
needed. \emph{Energy Policy}, \emph{49}, 143--148.
\url{https://doi.org/10.1016/j.enpol.2012.02.035}

\leavevmode\vadjust pre{\hypertarget{ref-Bouzarovski2018UnderstandingEP}{}}%
Bouzarovski, S., \& Bouzarovski, S. (2018). Understanding energy
poverty, vulnerability and justice. \emph{Energy Poverty: (Dis)
Assembling Europe's Infrastructural Divide}, 9--39.

\leavevmode\vadjust pre{\hypertarget{ref-Bouzarovski2016UnpackingTS}{}}%
Bouzarovski, S., Herrero, S. T., Petrova, S., \& Ürge-Vorsatz, D.
(2016). Unpacking the spaces and politics of energy poverty:
Path-dependencies, deprivation and fuel switching in post-communist
hungary. \emph{Local Environment}, \emph{21}, 1151--1170.
\url{https://api.semanticscholar.org/CorpusID:154156944}

\leavevmode\vadjust pre{\hypertarget{ref-Bouzarovski2015AGP}{}}%
Bouzarovski, S., \& Petrova, S. (2015). A global perspective on domestic
energy deprivation: Overcoming the energy poverty-fuel poverty binary.
\emph{Energy Research and Social Science}, \emph{10}, 31--40.

\leavevmode\vadjust pre{\hypertarget{ref-Bouzarovski2017SpatializingEJ}{}}%
Bouzarovski, S., \& Simcock, N. (2017). Spatializing energy justice.
\emph{Energy Policy}, \emph{107}, 640--648.

\leavevmode\vadjust pre{\hypertarget{ref-Brandon2024ASR}{}}%
Brandon, J. L., \& Bagnall, A. M. (2024). A systematic review to explore
factors affecting participation in health research trials amongst
underrepresented socio-economically disadvantaged populations in the UK
and ireland. \emph{medRxiv}.
\url{https://api.semanticscholar.org/CorpusID:273178429}

\leavevmode\vadjust pre{\hypertarget{ref-EnergyStatisticsHandbook2023}{}}%
Bureau of Energy, M. of E. A. (2024). \emph{Energy statistics handbook
2023}. Bureau of Energy, Ministry of Economic Affairs.
\url{https://www.moeaea.gov.tw/ECW_WEBPAGE/FlipBook/2023EnergyStaHandBook/index.html\#p=189}

\leavevmode\vadjust pre{\hypertarget{ref-Chan2023EnergyPA}{}}%
Chan, C., \& Delina, L. L. (2023). Energy poverty and beyond: The state,
contexts, and trajectories of energy poverty studies in asia.
\emph{Energy Research \&Amp; Social Science}.

\leavevmode\vadjust pre{\hypertarget{ref-Chen2015DeterminantsOP}{}}%
Chen, K.-M., \& Wang, T. (2015). Determinants of poverty status in
taiwan: A multilevel approach. \emph{Social Indicators Research},
\emph{123}, 371--389.

\leavevmode\vadjust pre{\hypertarget{ref-Chen2023PerspectivesOT}{}}%
Chen, P.-H., Lee, C., Wu, J., \& Chen, W.-S. (2023). Perspectives on
taiwan's pathway to net-zero emissions. \emph{Sustainability}.

\leavevmode\vadjust pre{\hypertarget{ref-Cheng2023TheCF}{}}%
Cheng, Y. (2023). The changing face of intimate premarital relationships
in taiwan. \emph{Journal of Marriage and Family}.

\leavevmode\vadjust pre{\hypertarget{ref-Chien2020PacingFR}{}}%
Chien, K. (2020). Pacing for renewable energy development: The
developmental state in taiwan's offshore wind power. \emph{Annals of the
American Association of Geographers}, \emph{110}, 793--807.

\leavevmode\vadjust pre{\hypertarget{ref-Chien2022AnIF}{}}%
Chien, K. (2022). An indigestible feast? A multi-scalar approach to the
energy transition in taiwan. \emph{Energy Research \& Social Science}.

\leavevmode\vadjust pre{\hypertarget{ref-Chou2023CarbonTI}{}}%
Chou, K.-T., \& Liou, H. (2023). Carbon tax in taiwan: Path dependence
and the high-carbon regime. \emph{Energies}.

\leavevmode\vadjust pre{\hypertarget{ref-Das2022QuantifyingTP}{}}%
Das, R. R., Martiskainen, M., \& Li, G. (2022). Quantifying the
prevalence of energy poverty across canada: Estimating domestic energy
burden using an expenditures approach. \emph{The Canadian Geographer /
Le Géographe Canadien}.

\leavevmode\vadjust pre{\hypertarget{ref-DGBAS2022}{}}%
DGBAS. (2022). \emph{The survey of family income and expenditure, 2021
(AA170046) {[}data file{]}} {[}Data set{]}.
\url{https://doi.org/10.6141/TW-SRDA-AA170046-1}

\leavevmode\vadjust pre{\hypertarget{ref-DGBAS2024}{}}%
DGBAS. (2024). \emph{The survey of family income and expenditure, 2023
(AA170048) {[}data file{]}} {[}Data set{]}.
\url{https://doi.org/10.6141/TW-SRDA-AA170048-1}

\leavevmode\vadjust pre{\hypertarget{ref-Gonzuxe1lezEguino2015EnergyPA}{}}%
González-Eguino, M. (2015). Energy poverty: An overview. \emph{Renewable
\& Sustainable Energy Reviews}, \emph{47}, 377--385.

\leavevmode\vadjust pre{\hypertarget{ref-Guan2023BurdenOT}{}}%
Guan, Y., Yan, J., Shan, Y., Zhou, Y., Hang, Y., Li, R., Liu, Y., Liu,
B., Nie, Q., Bruckner, B., Feng, K., \& Hubacek, K. (2023). Burden of
the global energy price crisis on households. \emph{Nature Energy},
\emph{8}, 304--316.

\leavevmode\vadjust pre{\hypertarget{ref-Guevara2022TheEO}{}}%
Guevara, Z., Espinosa, M., \& López-Corona, O. (2022). \emph{The
evolution of energy poverty theory: A scientometrics approach}.

\leavevmode\vadjust pre{\hypertarget{ref-Hafner2020TheGO}{}}%
Hafner, M., \& Tagliapietra, S. (2020). \emph{The geopolitics of the
global energy transition}.

\leavevmode\vadjust pre{\hypertarget{ref-Hagenaars1994PovertySI}{}}%
Hagenaars, A. J. M., Vos, K., \& Zaidi, M. (1994). \emph{Poverty
statistics in the late 1980s: Research based on micro-data}.

\leavevmode\vadjust pre{\hypertarget{ref-Heindl2013MeasuringFP}{}}%
Heindl, P. (2013). \emph{Measuring fuel poverty: General considerations
and application to german household data}.

\leavevmode\vadjust pre{\hypertarget{ref-Herrejuxf3n2023LivingWE}{}}%
Herrejón, P. V., Lennon, B., \& Dunphy, N. (2023). \emph{Living with
energy poverty}.

\leavevmode\vadjust pre{\hypertarget{ref-Herrero2017EnergyPI}{}}%
Herrero, S. (2017). Energy poverty indicators: A critical review of
methods. \emph{Indoor and Built Environment}, \emph{26}, 1018--1031.

\leavevmode\vadjust pre{\hypertarget{ref-Hills2012GettingTM}{}}%
Hills, J. (2012). \emph{Getting the measure of fuel poverty: Final
report of the fuel poverty review}.

\leavevmode\vadjust pre{\hypertarget{ref-Hostettler2015EnergyCI}{}}%
Hostettler, S. (2015). \emph{Energy challenges in the global south}.
3--9.

\leavevmode\vadjust pre{\hypertarget{ref-Hox2017MultilevelA}{}}%
Hox, J., Moerbeek, M., \& Schoot, R. (2017). \emph{Multilevel analysis :
Techniques and applications, third edition}.

\leavevmode\vadjust pre{\hypertarget{ref-Huang2015TheDO}{}}%
Huang, W.-H. (2015). The determinants of household electricity
consumption in taiwan: Evidence from quantile regression. \emph{Energy},
\emph{87}, 120--133.

\leavevmode\vadjust pre{\hypertarget{ref-Isherwood1979}{}}%
Isherwood, R. M., \& Hancock, B. C. (1979). \emph{Household expenditure
on fuel: Distributional aspects}. Economic Adviser's Office, DHSS.

\leavevmode\vadjust pre{\hypertarget{ref-Jessel2019EnergyPA}{}}%
Jessel, S., Sawyer, S., \& Hernández, D. (2019). Energy, poverty, and
health in climate change: A comprehensive review of an emerging
literature. \emph{Frontiers in Public Health}, \emph{7}.

\leavevmode\vadjust pre{\hypertarget{ref-Kemp-Benedict2014ShiftingTA}{}}%
Kemp-Benedict, E. (2014). \emph{Shifting to a green economy: Lock-in,
path dependence, and policy options}.

\leavevmode\vadjust pre{\hypertarget{ref-Kumar2019EnergyPI}{}}%
Kumar, P., Rao, S., \& Yadama, G. (2019). Energy poverty in india.
\emph{Encyclopedia of Social Work}.

\leavevmode\vadjust pre{\hypertarget{ref-Li2014EnergyPO}{}}%
Li, K., Lloyd, B., Liang, X.-J., \& Wei, Y.-M. (2014). Energy poor or
fuel poor: What are the differences? \emph{Energy Policy}, \emph{68},
476--481.

\leavevmode\vadjust pre{\hypertarget{ref-Lin2020NationalET}{}}%
Lin, M., Liou, H., \& Chou, K.-T. (2020). National energy transition
framework toward SDG7 with legal reforms and policy bundles: The case of
taiwan and its comparison with japan. \emph{Energies}.

\leavevmode\vadjust pre{\hypertarget{ref-Liu2023SpilloverEA}{}}%
Liu, D., Liu, X., Guo, K., Ji, Q., \& Chang, Y. (2023). Spillover
effects among electricity prices, traditional energy prices and carbon
market under climate risk. \emph{International Journal of Environmental
Research and Public Health}, \emph{20}.

\leavevmode\vadjust pre{\hypertarget{ref-Longe2021AnAO}{}}%
Longe, O. M. (2021). An assessment of the energy poverty and gender
nexus towards clean energy adoption in rural south africa.
\emph{Energies}.

\leavevmode\vadjust pre{\hypertarget{ref-Mastrucci2019ImprovingTS}{}}%
Mastrucci, A., Byers, E., Pachauri, S., \& Rao, N. D. (2019). Improving
the SDG energy poverty targets: Residential cooling needs in the global
south. \emph{Energy and Buildings}.

\leavevmode\vadjust pre{\hypertarget{ref-Middlemiss2022WhoIV}{}}%
Middlemiss, L. (2022). Who is vulnerable to energy poverty in the global
north, and what is their experience? \emph{Wiley Interdisciplinary
Reviews: Energy and Environment}, \emph{11}.

\leavevmode\vadjust pre{\hypertarget{ref-Monyei2018ExaminingES}{}}%
Monyei, C., Jenkins, K. E. H., Serestina, V., \& Adewumi, A. (2018).
Examining energy sufficiency and energy mobility in the global south
through the energy justice framework. \emph{Energy Policy}, \emph{119},
68--76.

\leavevmode\vadjust pre{\hypertarget{ref-Moore2012DefinitionsOF}{}}%
Moore, R. (2012). Definitions of fuel poverty: Implications for policy.
\emph{Energy Policy}, \emph{49}, 19--26.

\leavevmode\vadjust pre{\hypertarget{ref-ndc_2024}{}}%
National Development Council. (2024). \emph{Taiwan's pathway to net-zero
emissions in 2050}.
\url{https://www.ndc.gov.tw/en/Content_List.aspx?n=B154724D802DC488}

\leavevmode\vadjust pre{\hypertarget{ref-Niez2010ComparativeSO}{}}%
Niez, A. (2010). \emph{Comparative study on rural electrification
policies in emerging economies: Keys to successful policies}.

\leavevmode\vadjust pre{\hypertarget{ref-Njiru2018EnergyPA}{}}%
Njiru, C., \& Letema, S. (2018). Energy poverty and its implication on
standard of living in kirinyaga, kenya. \emph{Journal of Energy}.

\leavevmode\vadjust pre{\hypertarget{ref-Phimister2015TheDO}{}}%
Phimister, E., Vera-Toscano, E., \& Roberts, D. (2015). The dynamics of
energy poverty: Evidence from spain. \emph{Economics of Energy and
Environmental Policy}, \emph{4}, 153--166.

\leavevmode\vadjust pre{\hypertarget{ref-Pien2023TheDS}{}}%
Pien, C., Chao, C., \& Chou, K. (2023). The developmental state's legacy
and corporate carbon emission performance: Evidence from taiwanese firms
between 2014 and 2018. \emph{Climate and Development}.
\url{https://api.semanticscholar.org/CorpusID:265598091}

\leavevmode\vadjust pre{\hypertarget{ref-Riva2023EnergyPA}{}}%
Riva, M., Makasi, S. K., O'Sullivan, K., Das, R. R., Dufresne, P.,
Kaiser, D., \& Breau, S. (2023). Energy poverty: An overlooked
determinant of health and climate resilience in canada. \emph{Canadian
Journal of Public Health = Revue Canadienne de Santé Publique},
\emph{114}, 422--431.

\leavevmode\vadjust pre{\hypertarget{ref-Robinson2019EnergyPA}{}}%
Robinson, C. (2019). Energy poverty and gender in england: A spatial
perspective. \emph{Geoforum}.

\leavevmode\vadjust pre{\hypertarget{ref-Samarakoon2019AJA}{}}%
Samarakoon, S. (2019). A justice and wellbeing centered framework for
analysing energy poverty in the global south. \emph{Ecological
Economics}.

\leavevmode\vadjust pre{\hypertarget{ref-Suxe1nchez-Guevara2019AssessingPV}{}}%
Sánchez-Guevara, C., Peiró, M. N., Taylor, J., Mavrogianni, A., \&
González, J. N. (2019). Assessing population vulnerability towards
summer energy poverty: Case studies of madrid and london. \emph{Energy
and Buildings}.

\leavevmode\vadjust pre{\hypertarget{ref-Schuessler2014EnergyPI}{}}%
Schuessler, R. (2014). Energy poverty indicators: Conceptual issues -
part i: The ten-percent-rule and double median/mean indicators.
\emph{ERN: Environmental Studies (Topic)}.

\leavevmode\vadjust pre{\hypertarget{ref-Sen1976PovertyAO}{}}%
Sen, A. (1976). Poverty: An ordinal approach to measurement.
\emph{Econometrica}, \emph{44}, 219--231.

\leavevmode\vadjust pre{\hypertarget{ref-Sharma2023PovertyAG}{}}%
Sharma, M. (2023). Poverty and gender: Determinants of female- and
male-headed households with children in poverty in the USA, 2019.
\emph{Sustainability}.

\leavevmode\vadjust pre{\hypertarget{ref-Su2019ResidentialED}{}}%
Su, Y.-W. (2019). Residential electricity demand in taiwan: Consumption
behavior and rebound effect. \emph{Energy Policy}.

\leavevmode\vadjust pre{\hypertarget{ref-Su2020ResidentialED}{}}%
Su, Y.-W. (2020). Residential electricity demand in taiwan: The effects
of urbanization and energy poverty. \emph{Journal of the Asia Pacific
Economy}, \emph{25}, 733--756.

\leavevmode\vadjust pre{\hypertarget{ref-TRI2023}{}}%
Taiwan Research Institute. (2023).
\emph{112年下半年公用售電業電價費率檢討方案編製說明}.
\url{https://ele.tri.org.tw/1120919-1/1120919-1-2/mobile/index.html}

\leavevmode\vadjust pre{\hypertarget{ref-Thomson2017RethinkingTM}{}}%
Thomson, H., Bouzarovski, S., \& Snell, C. (2017). Rethinking the
measurement of energy poverty in europe: A critical analysis of
indicators and data. \emph{Indoor + Built Environment}, \emph{26},
879--901.

\leavevmode\vadjust pre{\hypertarget{ref-Tuttle2017TheEO}{}}%
Tuttle, C. J., \& Beatty, T. K. M. (2017). The effects of energy price
shocks on household food security in low-income households.
\emph{Economic Research Report}.

\leavevmode\vadjust pre{\hypertarget{ref-Zapata-Webborn2024WinterDF}{}}%
Zapata-Webborn, E., Hanmer, C., Oreszczyn, T., Huebner, G., McKenna, E.,
Few, J., Elam, S., Pullinger, M., Cheshire, C., Friel, D., Masters, H.,
\& Whittaker, A. (2024). Winter demand falls as fuel bills rise:
Understanding the energy impacts of the cost-of-living crisis on british
households. \emph{Energy and Buildings}.

\end{CSLReferences}

\appendix{A}{R code}

\appendix{B}{Data Visualization}

\end{document}
